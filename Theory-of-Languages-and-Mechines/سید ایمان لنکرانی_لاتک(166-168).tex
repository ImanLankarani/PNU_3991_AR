\documentclass[a4,12pt]{book}

\usepackage{xcolor}
\usepackage{graphicx}

\usepackage{color}
\usepackage{tikz}


\begin{document}

\begin{flushleft}
\textbf{166}\hspace*{1cm} \texttt{CHAPTER ELEVEN}
\end{flushleft}

\vspace*{1cm}
this solution may be problematic as some do not allow external cookies to be stored on their machines. For these users, a choice of response indicating the user has already completed or refused to complete the survey may be useful to delete these users from response-rate calculations. Together these techniques allowed Dahlen to conduct a Web survey on a major Swedish site that collected over 2,600 responses in six days by issuing an invitation to a randomly selected sample of one in every 200 unique vis-itors to the site. The number of respondents who explicitly refused to participate was a low 12 percent. This example illustrates the ways in which enhanced Web technolo-gies (e.g., server-side programs, cookies, etc.) can be used to overcome some of the challenges of sample selection that are associated with Web-based surveys. We can expect such programming enhancements to continue to create more powerful tools to aid the e-researcher.\\

\vspace*{0.5cm}

\textbf{Strategies for Attracting General Respondents to Your Web-Based Survey}\\

\vspace*{0.2cm}
In many types of survey research the target and sample populations are well known and personal communications are the most effective way of soliciting respondents. The hints provided earlier for obtaining email addresses or mailing postal invitations to par-ticipate in a Web survey apply when the sample audience is well known. Providing a hot link within the invitational email is the most common way to solicit participation in Web-based surveys.\\

\hspace*{0.5cm} Some types of w-research design seek large numbers of respondents from more generalized populations, for example, parents, taxpayers, or other large groups of potential respondents. The Web provides means to reach very large numbers of poten-tial respondents; and thus it is possible to get rather large numbers of respondents using these ''broadcast'' type appeals. It is tempting to think that respondents, especially when they are numerous, are representative of a larger or the whole popu-lation. However, the cautious e-researcher knows that population is self-selected, and, though their opinions may be interesting and useful, they are not representative of any particular population. To attract and induce these larger samples a variety of promotional and awareness tactics can be used. Following are a few tactics that the e-researcher can use.\\

\vspace*{0.5cm}

\textbf{Register with the Major Search Engines.}  \hspace*{0.2cm} A number of sites provide tutorials (i.e.,http://www.citiescommerce.com/consult.htm) on ways to promote the site that con-tains your e-research survey and automated programs (i.e., http://selfpromotion.com/) that allow researchers to list their site with multiple search engines.\\

 \vspace*{0.4cm}
\textbf{Obtain Links from Related Sites.} \hspace*{0.2cm} Since e-researchers are often interested in a par-ticular subset of the general population (e.g., English language teachers or disabled students), appropriate subjects can often be obtained by linking from major, well-trafficked Web sites that are frequented by members of the target population. The eas-iest way to obtain these links is to write a carefully crafted email to the site owner requesting that a link to your site be created. A small thumbnail icon for use as a link\\

\newpage

\begin{flushright}
 \texttt{SURVEYS} \hspace*{1cm} \textbf{167}
\end{flushright}

\vspace*{0.5cm}
may be appreciated by the Web site ower. Often Web site owners are interested in supporting qualified research related to the focus of their site.\\

 \vspace*{0.4cm}
 \textbf{Paid Banner Adventisements. }  \hspace*{0.2cm} Banner ads are the most prevalent form of commer-cial promotion currently used on the Web. Such banners can be purchased by the e-researcher to promote the site and the Web-based survey. However, the economical researcher should first assert whether many potential and appropriate respondents regularly frequent the site. In addition, the researcher will want to negotiate the price for such services, as there seems to be single means (number of viewings, number of click-throughs or individuals who actually use the Web banner to link to the target site, number of successfully completed referrals, etc.) nor a standardized price for maintenance of a Web banner on a commercial site. Interneted e-researchers should refer to http://www.wilsonweb.com/webmarket/ad-pricing.htm for discussion and links to current pricing models.\\

  \vspace*{0.4cm}

\textbf{Post to Appropriate Email Lists or Usenet Groups.}  \hspace*{0.2cm} Since many potential respon-dents will already be members of affiliated mailing lists, posting requests to participate in related email lists or Usenet groups is an obvious and very inexpensive means to attract e-survey respondents.However, e-researchers should be cognizant of both offi-cial policy and the unofficial culture of such groups in regard to unsolicited postings.
Generally, solicitations for research of a noncommercial nature are an acceptable use of appropriate mailing lists and Usenet groups. Appropriate groups are those whose members are generally interested in the subject of the e-research. For example a sur-vey seeking teacher respondents related to experiences of online collaborative writing projects is an appropriate posting in alt.education.alternative or to the list ECOMP-L College English Composition Discussion List, but would be inappropriate for posting to lists or newsgroups relating to dog breeding or e-commerce.\\

   \vspace*{0.4cm}


\textbf{Provide Incentives.} \hspace*{0.2cm} Perception of reward is a major factor in respondent comple-tion of survey research. In the most up-front use of incentives, respondents are paid directly for completion of the survey (see http://momoneyclues.webhostme.com/sur-veys.htm for a listing of firms that pay online, mail, and telephone survey respon-dents). Research on paper-based incentives shows that immediate rewards are more effective than promised rewards in the future (Church, 1993).For example, including an electronic gift certificate for a popular online vendor is more  effective than promis-ing a check to be delivered by mail some months later. Nonmonetary incentives such as promises of recognition, copies of final e-research results, and e-lottery tickets may also increase participation rates. However there is little solid research confirming cost/benefit rations of such incentives for either the respondent.\\

 \vspace*{0.3cm}
\textbf{Advertise in Traditional Media.}  \hspace*{0.2cm} Traditional media includes such communication formats as newspapers, posters, telephone, and the like. Placing ads in traditional media may be an appropriate way to reach potential respondents. The media used should be related to the interest of the target population to obtain good returns how-ever the size of circulation and cost per insertion must also be considered when using traditional media to advertise e-research opportunities.\\


\newpage
\begin{flushleft}
\textbf{168}\hspace*{1cm} \texttt{CHAPTER ELEVEN}
\end{flushleft}

\vspace*{0.5cm}
\begin{flushleft}
TABLE 10.1 Comparison of Email And Web-Based E-Surveys
\begin{tabular}{ccc}

 \hline

 \hline

 \hline

 \hline

 $EMAIL$ &  $SURVEYS$ & $WEB-BASED SURVEYS$\\
\hline
Advantages  &  Pushed to subject's private mailbox  &  Easy error checking \\
            &  Ubiquity of email                    &  Instant results \\
            &  Guarantee of privacy                 &  Monitoring of subject behavior \\
            &  Can be printed and returned          &  \hspace*{0.2cm} while completing survey\\
            &  \hspace*{0.2cm} via post, fax,or email&                                       \\
  \vspace*{0.2cm}
Challenges  &  More difficult to error check        &  Users need to be pulled to\\
            &  Results must be parsed from          &  \hspace*{0.2cm} the site \\
            &  \hspace*{0.2cm} returned email       &  Constraints on anonymity \\
            &  No anonymity unless returned by      &                           \\
            &  \hspace*{0.2cm} post or fax or a ''stealth service'' &           \\
 \hline

 \hline

 \hline

 \hline
\end{tabular}
\end{flushleft}

\vspace*{0.5cm}

\hspace*{0.5cm} Finally, is email a better alternative than Web-based surveying? Table 11.1 illus-trates a comparison of the advantages and challenges of email surveys versus Web-based surveys.\\

\vspace*{0.5cm}
\textbf{COMMERCIAL e-SURVEY PACKAGES}\\

\vspace*{0.2cm}
\small{
Like the creation of home pages, Net-based surveys can bee created using the simplest of text editors. However, many people find that the purchase or rental of dedicated software makes the task easier and faster, while adding features that would require pro-gramming expertise beyond that of most e-researchers. Survey packages originated as ways to create, analyze, and organize paper-and-pencil surveys. Many packages are now Net enabled allowing for creation, distribution, and analysis of surveys via the Web, email, fax, or post. The development and increasing sophistication of these prod-ucts will continue past the date of this text, so it is useful to check with current home-pages of the manufacturers and reviewing guides, such as those printed in \emph{PCW orld} for the latest software developments. Next we discuss the features of the most popular e-survey software currently available.}\\

\vspace*{0.5cm}

\textbf{FEATURES OF POPULAR SURVEY PACKAGES}\\

\vspace*{0.5cm}
\textbf{Survey Creation}\\

\vspace*{0.3cm}
A good software package aids in the creation of surveys in a number of useful ways.Many packages provide ''wizards'' that present fill-in forms that the researcher com-pletes and then the program automatically formats the survey data. Some packages provide examples of generic question types (multiple choice, matching, etc.) that pro-vide structural and formatting guidance to the creator as he or she edits these questions to reflect his or her owe content. Some packages also support database storage of \\

\end{document}  